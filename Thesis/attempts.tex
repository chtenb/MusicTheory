%        File: HarmonyTheory.tex
%     Created: Sat Sep 08 08:00 PM 2012 C
% Last Change: Sat Sep 08 08:00 PM 2012 C
%
\documentclass[a4paper]{article}
\usepackage{amssymb,amsmath,amsthm,enumerate}

\title{Music Theory}
\author{Chiel ten Brinke}

\newtheorem{theorem}{Theorem}[section]
\newtheorem{lemma}[theorem]{Lemma}
\newtheorem{proposition}[theorem]{Proposition}
\newtheorem{corollary}[theorem]{Corollary}

\theoremstyle{definition}
\newtheorem{definition}[theorem]{Definition}
\newtheorem{example}[theorem]{Example}
\newtheorem{remark}[theorem]{Remark}
\newtheorem{procedure}[theorem]{Procedure}

\begin{document}
\maketitle

\newcommand{\Q}{\mathbb{Q}_{> 0}}
\newcommand{\R}{\mathbb{R}_{> 0}}
%\newcommand{\T}{\mathbb{T}}
\newcommand{\Primes}{\mathbb{P}}
\newcommand{\I}{\mathbb{I}}

\section{Pure tonal space}

A few notation conventions:
\begin{enumerate}[i]
	\item 
		The ordinary multiplication between numbers is denoted by $\cdot$.
	\item 
		When $G$ is a set and $\cdot$ an operation defined on $G$, we write the group generated by $G$ as $(G, \cdot)$.
	\item 
		When $(G, \cdot)$ is a group, we denote the group by $G$ as well.
\end{enumerate}


\begin{definition}
	A \emph{pure tonal space} $T$ is a subgroup of $(\Q, \cdot)$
	A \emph{pure tone} is an element of a pure tonal space.
\end{definition}

Notice that the ordinary number multiplication makes a tonal space abelian.

\begin{proposition}
	$(\Q, \cdot) = (\Primes,\cdot) = (\I,\cdot)$, where $\Primes$ is the set of all primes, and $\I$ is the set of all numbers of the form $\frac{n+1}{n}$, $n \in \mathbb{N}$.
\end{proposition}

\section{Metrics on pure tonal spaces}
We can turn a pure tonal space into a metric space, by defining a metric on it.

\begin{definition}
	A \emph{norm} on a pure tonal space $T$ is a function $|\cdot|:T \to [0,\infty)$, sufficing the following properties:
	\begin{enumerate}[i]
		\item $|t| = 0 \Leftrightarrow t=1$
		\item $|t| = |\frac{1}{t}|$
		\item $|t \cdot s| \leq |t| + |s|$
	\end{enumerate}
\end{definition}

\begin{definition}
	A \emph{metric} on a pure tonal space $T$ is a function $d:T \times T \to [0,\infty)$, sufficing the following properties:
	\begin{enumerate}[i]
		\item $d(s,t) = 0 \Leftrightarrow s=t$
		\item $d(s,t) = d(t,s)$
		\item $d(r,t) \leq d(r,s) + d(s,t)$
	\end{enumerate}
\end{definition}

\begin{remark}
	Note that if a norm on a pure tonal space $T$ is given, one can deduce a metric on $T$ from this.
	Define $d(s,t) = |\frac{s}{t}|$.
	We show that $d$ suffices the properties of a metric.
	Notice that $d(s,t) = 0 \Leftrightarrow |\frac{s}{t}| = 0 \Leftrightarrow \frac{s}{t} = 1 \Leftrightarrow s = t$.
	We also see that $d(s,t) = |\frac{s}{t}| = |\frac{1}{s/t}| = |\frac{t}{s}| = d(t,s)$.
	Finally note that $d(r,t) = |\frac{r}{t}| = |\frac{r \cdot s}{s \cdot t}| \leq |\frac{r}{s}|+ |\frac{s}{t}| = d(r,s) + d(s,t)$.
	Thus $d$ is a metric on $T$.
	\label{norm_to_metric}
\end{remark}

We will interpret the distance between two tones as a measure of how hard their relation is to perceive.
Euler came up with the following metric.

\newcommand{\eulernorm}[1]{|#1|_{Euler}}
\newcommand{\eulermetric}{d_{Euler}}
\begin{example}
	Let $T$ be a pure tonal space.
	Define $\eulernorm{\cdot} : T \to [0,\infty )$ as follows.
	Let $p_0,\dots,p_n$ be the prime generators of $T$, sorted by magnitude.
	Then any $t \in T$ can be written as $p_0^{m_0}\cdot \dots \cdot p_n^{m_n}$, where $m_i \in \mathbb{Z}, 0 \leq i \leq n $.
	Define $\eulernorm{t} = |m_0| (p_0 - 1) + \dots + |m_n| (p_n - 1)$.
	Now we show that $\eulernorm{\cdot}$ is indeed a norm.
	Note that $\eulernorm{t} = 0 \Leftrightarrow m_i = 0, 0 \leq i \leq n \Leftrightarrow t = 1$.
	We also see that $\eulernorm{t} = |m_0| (p_0 - 1) + \dots + |m_n| (p_n - 1) = |{-m_0}| (p_0 - 1) + \dots + |{-m_n}| (p_n - 1) = |\frac{1}{t}|_E$.
	Finally, notice that $\eulernorm{t \cdot s} = \dots$.
	Thus $\eulernorm{\cdot}$ is a norm and induces the metric $\eulermetric$, as follows by remark \ref{norm_to_metric}.
\end{example}

In fact the above example is a generalization of Euler's actual metric.
Euler only defined it for natural numbers, but since we are working with a group, it makes sense to also define it for rational numbers.

We could visualize a pure tonal space by making the distinct prime generators the dimensions.

%Visualization here

This visualization suggests another metric, namely the euclidean metric.

\newcommand{\euclidnorm}[1]{|#1|_{Euclid}}
\newcommand{\euclidmetric}{d_{Euclid}}
\begin{example}
	Let $T$ be a pure tonal space.
	Let $p_0,\dots,p_n$ be the prime generators of $T$, sorted by magnitude.
	Then any $t \in T$ can be written as $p_0^{m_0}\cdot \dots \cdot p_n^{m_n}$, where $m_i \in \mathbb{Z}, 0 \leq i \leq n $.
	Define $\euclidnorm{t} = \sqrt{(m_0(p_0-1))^2 + \dots + (m_n(p_n-1))^2}$.
\end{example}

The given norms are only defined for pure tonal spaces.

\section{The general concept of tonal space}

\begin{definition}
	A \emph{tonal space} is a subgroup of $(\R,\cdot)$.
	A \emph{tone} $t$ is an element of a tonal space.
\end{definition}

\begin{example}
	Let $T$ be the cyclic group generated by $2^\frac{1}{12}$.
	This tonal space is informally known as the twelve-tone equal-tempered scale.
\end{example}

% Remark about existing scales, such as just intonation, that are not a group.
% They are in fact a subset of a tonal space

\section{Approximating pure tonal space}
A pure tonal space can be approximated by a cyclic tonal space, thanks to the tolerance of the ear.
The generator has to be chosen carefully, such that any important interval can be approximated by powers of the generator.

\begin{definition}
Let $A \subset \R$ and $\delta > 1$.
Now let $T$ be a cyclic tonal space, such that any $a \in A$ is \emph{covered by $T$}, i.e. there is a $t \in T$ such that $\frac{1}{\delta}t \leq a \leq \delta t$.
Then $T$ is said to be a \emph{cyclic $\delta$-approximation of $A$}.
\end{definition}

%We present a procedure that determines the cyclic $\delta$-approximation of a finite $A \subset \T$ with the greatest generator, given $\delta$ and $A$.
%\begin{procedure}
%	Let $A \subset \T$ be finite and $\delta > 1$.
%	Note that any cyclic group has at least two potential generators, namely the known generator and its inverse.
%	For tonal spaces, this means that there is a generator smaller than one, and a generator greater than one. (Lemma?)
%	$A$ only has to contain number strictly greater than 1, since we can substitute any $a<1$ with its inverse, and we can disregard 1. (Lemma?)
%	We look for the cyclic $\delta$-approximation of $A$ named $T$, with maximal generator $t > 1$.
%	If $t$ has been found, $T$ follows, since $t$ is the generator.
%	The essence of the procedure is to obtain an upper bound for $t$ and tighten this bound inductively.
%
%	Suppose there exists an upper bound $x$ for $t$.
%	Choose $a \in A$ that is not covered by the tonal space generated by $x$.
%	If no such $a$ exists, we are done for $t=x$, so assume it exists.
%	Consider the smallest $n$ such that $\delta a \leq x^n$.
%	Let $d = \sqrt[n]{x^n/\delta a}$.
%	Note that $d$ can be seen as the number $x$ has to be divided by to make $a$ be near enough to some power of $x$.
%	Indeed $(x/d)^n = \frac{x^n}{x^n/\delta a} = \delta a$.
%	This means, that $x/d$ is a tighter upper bound for $t$.
%	Now repeat this step for $x = x/d$.
%
%	Now we only need an initial upper bound.
%	Choose $a \in A$. Let $x = \delta a$.
%	Now $x$ is an upperbound for $t$.
%
%	\label{finite_approximation_procedure}
%\end{procedure}
%
%\begin{theorem}
%	Procedure~\ref{finite_approximation_procedure} converges and gives the correct result.
%\end{theorem}

If $X \subset \R$ is infinite, we can compute an approximation $A$ of $X$, provided that $X$ is periodic, i.e. there exists an $p$ such that every $x \in X$ can be written as $px'$, where $1 \leq x' \leq p$.
In fact this is what makes the algorithm easy, since $p$ must be an element of $A$.
This means that we will only have to consider roots of $p$ as generators.
\begin{procedure}
	Let $X$ as above.
	Now let $n=1$.
	If every $x \in X$ is covered by the group generated by $p^\frac{1}{n}$, we are done.
	If not, repeat the above for $n=n+1$.
	\label{infinite_approximation_procedure}
\end{procedure}

\begin{theorem}
	Procedure~\ref{infinite_approximation_procedure} terminates and gives the correct result.
\end{theorem}

The other way around, if we have an approximation $A$ of $X \subset \R$, we can determine the least $\delta$ for $A$ to be a correct approximation.
\begin{procedure}
	Let $X \subset \R$ be infinite, $p \in X$ such that every $x \in X$ can be written as $px'$, where $1 \leq x' \leq p$ and let $A$ be a cyclic approximation.
	Now let $\delta' = \max_{x \in X}(\min_{a \in A}(\frac{a}{x}))$.
	Finally $\delta = \left\{
		\begin{array}{ll}
			\delta' 		& if \delta' > 1\\ 
			\frac{1}{\delta'} 	& otherwise
		\end{array} 
		\right.$
	\label{delta_determination_procedure}
\end{procedure}

\begin{theorem}
	Procedure~\ref{delta_determination_procedure} terminates and gives the correct result.
\end{theorem} 

\begin{remark}
	Given a tone $t$ and an approximation $A$ with generator $a$, we can easily map the tone onto the approximation.
	The power of the approximation of $t$ is given by $\mathrm{round}(^a\log(t))$, and thus the actual tone is given by $a^{\mathrm{round}(^a\log(t))}$.
\end{remark}

\end{document}
