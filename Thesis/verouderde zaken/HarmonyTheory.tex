%        File: Dropbox/MuziekTheorie/Harmony Theory.tex
%     Created: Mon Apr 02 05:00 PM 2012 C
% Last Change: Mon Apr 02 05:00 PM 2012 C
%
\documentclass[a4paper]{report}
\usepackage{amssymb,amsmath,amsthm}
\usepackage[dutch]{babel}

\author{Chiel ten Brinke \\ Roelof ten Napel}
\title{Fundamental Harmony Theory}

\newtheorem{thm}{Theorem}
\newtheorem{dfn}[thm]{Definition}

\newcommand{\cln}{\mathbin{:}}

\begin{document}
\maketitle

For physical reasons we consider a tone as a wave.
We denote a tone by a number, which we can interpret as frequency.

\begin{dfn}
	A tone is denoted by a rational number.
	%If $t$ is a tone, another tone $s$ having a frequency $n \in \mathbb{N}$ times the frequency of $t$ is denoted by the number resulting from $nt$.
\end{dfn}

\begin{dfn}
	A harmony is a finite set of tones.
	If $H$ is a harmony consisting of tones $a_0,a_1,a_2,\dots$, we usually write $H = a_0 \cln a_1 \cln a_2 \cln \dots$.
\end{dfn}

\paragraph{Metaphysical point of view on perception of tones and harmonies}

As there is no absolute highest or lowest tone, we regard tones relative to other tones.
We therefore study the relationships between tones.
As Pythagoras and Euler have argued, the relationship of two tones is perceived as the ratio existing between them.
The more complex that ratio is, the more difficult it is to perceive.

From some metaphysical points of view, consonance of a harmony can be expressed as complexity of the relation between the individual tones of the harmony.
Let $t$ a tone that we regard as a base tone for now.
If we consider harmonics of $t$ in relation to the base tone $t$, then the 1st harmonic is not complex at all.
The 2nd harmonic is more complex than the 1st harmonic.


\paragraph{Mathematical definitions on complexity of harmony}
For reasons of experience and metaphisical reasons, we define the complexity of a ratio as follows.
Denote the complexity as a function $c:\mathbb{H} \to \mathbb{N}$, where $\mathbb{H}$ is the set of all harmonies.
We consider the complexity of the interval $a \cln b$.
We can write the ratio between them as a product of primes to the power of integers.
More precisely, every ratio $\frac{a}{b}$ can be written uniquely (disregarding consecution) as $\prod_{i=0}^{n} p_i^{z_i}$, where $(p_i)$ are distinct primes and $(z_i)$ are nonzero integers.


\begin{dfn}
	Let $H = a \cln b$ for arbitrary tones $a$ and $b$.
	%: \mathbb{H} \to \mathbb{N}$ 
	Let $\prod_{i=0}^{n} p_i^{z_i}$ be the unique factorization of the ratio between $a$ and $b$.
	Then $c(H) = \sum_{i=0}^{n} {(p_i-1)|z_i|}$
\end{dfn}

Now we only need to define $c$ for harmonies consisting of more than two tones.


\paragraph{Error}
A problem arises when one regards tones with nearly the same frequency.
The human ear perceives it as if it were a very simple ratio (namely 1).
But theoretically speaking, the ratio can be very complex.
One needs to take the inaccuracy of the human ear into account.
If we can show that $(\text{Human ear perceives } a \cln b \text{ as being consonant}) \Rightarrow (\frac{a}{b} \text{ is close to an easy ratio})$, we solved this problem by blaming the inaccuracy of the human ear.






\end{document}


